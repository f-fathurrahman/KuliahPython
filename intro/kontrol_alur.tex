\section{Kontrol alur program}

\subsection{Percabangan: \texttt{if/elif/else}}
  
\begin{pyconcode}
>>> a = 2
>>> if a < 0:
...     print("a bernilai negatif")
... else:
...     print("a bernilai positif")
... 
\end{pyconcode}


\textbf{Indentasi sangat penting pada Python.}

Jika Anda kesulitan atau tidak nyaman menuliskan di konsol, Anda
dapat menuliskannya di file dengan ekstensi \texttt{*.py}.
Anda tidak perlu menuliskan \verb|>>>| atau \verb|...|.

Contoh lain
\begin{pyconcode}
>>> a = 10 
>>> if a == 1:
...     print(1)
... elif a == 2:
...     print(2)
... else:
...     print('A lot')
... 
A lot
\end{pyconcode}


\subsection{Perulangan: \texttt{for/range}}

Perulangan dengan indeks
\begin{pyconcode}
>>> for i in range(4):
...     print(i)
... 
0
1
2
3
\end{pyconcode}

Iterasi terhadap nilai:
\begin{pyconcode}
>>> for word in ('cool', 'powerful', 'readable'):
...     print('Python is %s' % word)
...
Python is cool
Python is powerful
Python is readable
\end{pyconcode}

\subsection{Perulangan: \texttt{while/break/continue}}

\begin{pyconcode}
>>> a = 3;
>>> while a < 10:
...     a = a + 1
... 
>>> a
10
\end{pyconcode}

\texttt{break} dapat digunakan untuk
keluar dari loop \texttt{for} atau \texttt{while}
\begin{pyconcode}
>>> a = 10;
>>> while a > 0:
...     a = a - 2
...     if a < 3: break
... 
>>> a
2
\end{pyconcode}


\texttt{continue} dapat digunakan untuk meneruskan ke
iterasi selanjutnya (biasanya untuk melangkaui suatu iterasi)
\begin{pyconcode}
>>> a = [1, 0, 2, 4]
>>> for element in a:
...     if element == 0:
...         continue
...     print(1.0/element)
...     
1.0
0.5
0.25
\end{pyconcode}


\subsection{Iterasi pada string, list, dan dictionary}

Kita dapat melakukan iterasi terhadap string, list dan dictionary
\begin{pyconcode}
>>> vowels = 'aeiouy'
>>> for i in 'powerful':
...     if i in vowels:
...         print(i),
...         
...         
o e u
\end{pyconcode}

Contoh
\begin{pyconcode}
>>> message = "Hello how are you?"
>>> message.split() # mengembalikan list
['Hello', 'how', 'are', 'you?']
>>> for word in message.split():
...     print word
...     
Hello
how
are
you?
\end{pyconcode}

Contoh:
\begin{pyconcode}
>>> words = ('cool', 'powerful', 'readable')
>>> for i in range(0, len(words)):
....     print(i, words[i])
....     
....     
0 cool
1 powerful
2 readable
\end{pyconcode}

Fungsi \texttt{enumerate} juga dapat digunakan
\begin{pyconcode}
>>> words = ('cool', 'powerful', 'readable')
>>> for index, item in enumerate(words):
...     print index, item
...     
0 cool
1 powerful
2 readable
\end{pyconcode}


Loop pada dictionary dengan menggunakan \texttt{items()}
\begin{pyconcode}
>>> d = {"a" : 1, "b" : 1.2, "c" : 1j}
>>> for key, val in d.items():
        print('Key: %s has value: %s' % (key, val)
Key: a has value: 1
Key: c has value: 1j
Key: b has value: 1.2
\end{pyconcode}