\section{Menggunakan modul Python}

Banyak fungsionalitas dari Python yang dikumpulkan dalam sebuat modul.
Pada praktikum ini dan selanjutnya, kita akan banyak menggunakan modul-modul
tersebut.
Modul Python ini ada yang sifatnya standard atau bawaan dan yang lainnya
harus dipasang terlebih dahulu sebagai suatu paket atau pustaka.
Modul pertama yang akan kita temui adalah modul \texttt{math}. Modul ini
adalah modul standard bawaan Python.

Modul ini dapat diakses dengan menggunakan perintah \texttt{import math}.
\begin{pyconcode}
>>> import math
>>> dir(math)
['__doc__', '__file__', '__loader__', '__name__', '__package__', '__spec__', 'acos', 'acosh', 'asin', 'asinh', 'atan', 'atan2', 'atanh', 'ceil', 'copysign', 'cos', 'cosh', 'degrees', 'e', 'erf', 'erfc', 'exp', 'expm1', 'fabs', 'factorial', 'floor', 'fmod', 'frexp', 'fsum', 'gamma', 'gcd', 'hypot', 'inf', 'isclose', 'isfinite', 'isinf', 'isnan', 'ldexp', 'lgamma', 'log', 'log10', 'log1p', 'log2', 'modf', 'nan', 'pi', 'pow', 'radians', 'sin', 'sinh', 'sqrt', 'tan', 'tanh', 'tau', 'trunc']
\end{pyconcode}

Perintah \texttt{dir(math)} digunakan untuk mengetahui simbol
(fungsi dan/atau konstanta) apa saja yang
didefinisikan dalam module \texttt{math}. Anda dapat menggunakan perintah
\texttt{help(math.log2)} misalnya, untuk membaca dokumentasi mengenai fungsi ini.

Contoh penggunaan modul \texttt{math}
\begin{pyconcode}
>>> math.sin(math.pi)
1.2246467991473532e-16
>>> math.sin(0.5*math.pi)
1.0
>>> math.sin(0.3*math.pi)
0.8090169943749475
>>> math.sin(math.pi/3)
0.8660254037844386
\end{pyconcode}

Sintaks alternatif berikut ini juga sering digunakan.
\begin{pythoncode}
from math import *
\end{pythoncode}
Dengan sintaks di atas, semua fungsi pada modul \texttt{math} akan diekspor ke
konteks global Python.
Jika kita hanya ingin mengimpor beberapa simbol saja dapat digunakan
sintaks seperti berikut ini.
\begin{pythoncode}
from math import pi, factorial
\end{pythoncode}


\begin{mdframed}[backgroundcolor=myframebg]
\textbf{Tugas}

Buatlah suatu program sederhana untuk menghitung akar-akar dari
persamaan kuadrat dengan input dari pengguna. Anda dapat menggunakan
fungsi input untuk membaca input dari pengguna. Gunakan fungsi
\texttt{cmath.sqrt} yang dapat menghitung akar kuadrat dari bilangan
negatif sebagai bilangan kompleks.
\end{mdframed}

