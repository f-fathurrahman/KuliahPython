\documentclass[a4paper,11pt]{extarticle}
\usepackage[a4paper]{geometry}
\geometry{verbose,tmargin=2cm,bmargin=2cm,lmargin=2cm,rmargin=2cm}

\usepackage{fontspec}
\setmonofont{FreeMono}

\setlength{\parindent}{0cm}
\setlength{\parskip}{0.5em}

\usepackage{textcomp}

\usepackage{hyperref}
\usepackage{url}
\usepackage{xcolor}

\usepackage{minted}
\newminted{pycon}{breaklines,fontsize=\small}
\newminted{python}{breaklines,fontsize=\small}
\newminted{text}{breaklines,fontsize=\small}

\definecolor{mintedbg}{rgb}{0.95,0.95,0.95}
\usepackage{mdframed}

\BeforeBeginEnvironment{minted}{\begin{mdframed}[backgroundcolor=mintedbg]}
\AfterEndEnvironment{minted}{\end{mdframed}}

\title{
MI3103 \\
Praktikum Antar Muka Komputer\\
Pengantar Pemrograman Python}
\author{Fadjar Fathurrahman}
\date{2018}

\begin{document}
\maketitle

\section{Tujuan}
\begin{itemize}
\item dapat membuat dan mengeksekusi program Python sederhana
\item mampu mengenal dan menggunakan tipe data dasar pada Python
\item mampu mengenal dan menggunakan kontrol program sederhana pada Python (percabangan dan
perulangan)
\item mampu membuat fungsi sederhana (subrutin) pada Python
\end{itemize}

\section{Perangkat lunak yang diperlukan}
\begin{itemize}
\item Linux OS
\item Distribusi Anaconda untuk Python 3
\item Browser
\item Editor teks seperti:
  \begin{itemize}
  \item \textsf{gedit} (biasanya sudah tersedia pada banyak distro Linux)
  \item \textsf{VSCode} (\url{https://code.visualstudio.com/})
  \item \textsf{Atom} (\url{https://atom.io/})
  \end{itemize}
\end{itemize}

%\section{Pendahuluan}
%Python adalah salah satu bahasa pemrograman paling populer.
%\url{http://www.scipy-lectures.org/language/first_steps.html}

Python Anaconda pada Ubuntu 18.04 di LabKom
\texttt{/opt/anaconda3}

\section{Mengenal interpreter Python (konsol Python)}

Terdapat banyak pilihan untuk berinteraksi dengan interpreter 
atau konsol Python:
\begin{itemize}
\item Konsol default Python, dapat dijalankan dengan mengetikkan
\texttt{python} pada terminal.
\item IPython, merupakan interpreter Python dengan berbagai fitur tambahan
seperti \textit{tab-completion} dan \textit{syntax highlighting}.
IPython 
\item Jupyter qtconsole
\item Jupyter notebook
\end{itemize}

Tampilan awal konsol Python default:
\begin{textcode}
Python 3.6.6 |Anaconda custom (64-bit)| (default, Jun 28 2018, 17:14:51) 
[GCC 7.2.0] on linux
Type "help", "copyright", "credits" or "license" for more information.
>>>
\end{textcode}

Tampilan awal IPython
\begin{textcode}
Python 3.6.6 |Anaconda custom (64-bit)| (default, Jun 28 2018, 17:14:51) 
Type 'copyright', 'credits' or 'license' for more information
IPython 6.4.0 -- An enhanced Interactive Python. Type '?' for help.

In [1]:
\end{textcode}

Dalam praktikum ini Anda sebaiknya menggunakan IPython karena lebih
mudah digunakan. Jika IPython belum terpasang, Anda dapat menggunakan
konsol Python biasa.

Anda dapat menggunakan interpreter ini untuk menjalankan kode Python.
Kita akan mulai dengan contoh sederhana yaitu operasi aritmatika
sederhana.
\begin{pyconcode}
>>> 2 + 3
5
>>> 2 / 3
0.6666666666666666
\end{pyconcode}

Anda juga dapat menggunakan variabel.
\begin{pyconcode}
>>> a = 3
>>> b = 4.1
>>> a*b
12.299999999999999
>>> a - b
-1.0999999999999996
>>> a/b
0.7317073170731708
\end{pyconcode}

Berbeda dengan C/C++, suatu variabel tidak perlu dideklarasikan atau
tipenya sebelum digunakan. Python akan mendeduksi tipe variabel berdasarkan
nilai awal yang diberikan.

Integer variables::
\begin{pyconcode}
>>> 1 + 1
2
>>> a = 4
\end{pyconcode}

floats ::
\begin{pyconcode}
>>> c = 2.1
\end{pyconcode}

Python juga mendukung bilangan kompleks.
\begin{pyconcode}
>>> a = 1.5 + 0.5j
>>> a.real
1.5
>>> a.imag
0.5
\end{pyconcode}

and booleans::
\begin{pyconcode}
>>> 3 > 4
False
>>> test = (3 > 4)
>>> test
False
>>> type(test)
<type 'bool'>
\end{pyconcode}


\subsection*{Tugas}

Carilah operator matematika yang didukung oleh Python.
Beberapa sumber berikut dapat Anda gunakan:
\begin{itemize}
\item \url{https://www.tutorialspoint.com/python/python_basic_operators.htm}
\item \url{https://en.wikibooks.org/wiki/Python_Programming/Basic_Math}
\end{itemize}


\section{Kode sumber Python}

Selain menggunakan konsol Python, kita dapat menuliskan program
yang kita buat dalam suatu file teks dengan ekstensi \texttt{.py}.
Anda dapat membuat file ini dengan editor yang mendukung Python, beberapa
diantaranya adalah:
\begin{itemize}
\item \textsf{Gedit} yang biasanya sudah terpasang di Ubuntu dan banyak distribusi
Linux
\item \textsf{Atom} yang merupakan editor teks multifungsi. Editor ini dapat
dijalankan baik di Windows, OSX, maupun Linux.
\item \textsf{Visual Studio Code}, mirip dengan \textsf{Atom}.
Meskipun dikembangkan oleh Microsoft, editor ini juga dapat dijalankan di Linux dan
OSX.
\item \textsf{Spyder} yang merupakan IDE (\textit{integrated development environment}) untuk
Python.
\end{itemize}

Misalkan file program Python Anda bernama \texttt{hitung.py}, maka Anda dapat
mejalankan program ini dari terminal dengan perintah berikut.
\begin{textcode}
python hitung.py
\end{textcode}

Komentar pada Python dimulai dengan tanda \verb|#|.

\section{Modul Python}

Banyak fungsionalitas dari Python yang dikumpulkan dalam sebuat modul.
Pada praktikum ini dan selanjutnya, kita akan banyak menggunakan modul-modul
tersebut.
Modul Python ini ada yang sifatnya standard atau bawaan dan yang lainnya
harus dipasang terlebih dahulu sebagai suatu paket atau pustaka.
Modul pertama yang akan kita temui adalah modul \texttt{math}. Modul ini
adalah modul standard bawaan Python.

Modul ini dapat diakses dengan menggunakan perintah \texttt{import math}.
\begin{pyconcode}
>>> import math
>>> dir(math)
['__doc__', '__file__', '__loader__', '__name__', '__package__', '__spec__', 'acos', 'acosh', 'asin', 'asinh', 'atan', 'atan2', 'atanh', 'ceil', 'copysign', 'cos', 'cosh', 'degrees', 'e', 'erf', 'erfc', 'exp', 'expm1', 'fabs', 'factorial', 'floor', 'fmod', 'frexp', 'fsum', 'gamma', 'gcd', 'hypot', 'inf', 'isclose', 'isfinite', 'isinf', 'isnan', 'ldexp', 'lgamma', 'log', 'log10', 'log1p', 'log2', 'modf', 'nan', 'pi', 'pow', 'radians', 'sin', 'sinh', 'sqrt', 'tan', 'tanh', 'tau', 'trunc']
\end{pyconcode}

Perintah \texttt{dir(math)} digunakan untuk mengetahui simbol
(fungsi dan/atau konstanta) apa saja yang
didefinisikan dalam module \texttt{math}. Anda dapat menggunakan perintah
\texttt{help(math.log2)} misalnya, untuk membaca dokumentasi mengenai fungsi ini.

Contoh penggunaan modul \texttt{math}
\begin{pyconcode}
>>> math.sin(math.pi)
1.2246467991473532e-16
>>> math.sin(0.5*math.pi)
1.0
>>> math.sin(0.3*math.pi)
0.8090169943749475
>>> math.sin(math.pi/3)
0.8660254037844386
\end{pyconcode}

Sintaks alternatif berikut ini juga sering digunakan.
\begin{pythoncode}
from math import *
\end{pythoncode}
Dengan sintaks di atas, semua fungsi pada modul \texttt{math} akan diekspor ke
konteks global Python.
Jika kita hanya ingin mengimpor beberapa simbol saja dapat digunakan
sintaks seperti berikut ini.
\begin{pythoncode}
from math import pi, factorial
\end{pythoncode}


\subsection*{Tugas}

Buatlah suatu program sederhana untuk menghitung akar-akar dari
persamaan kuadrat.



Fungsi \texttt{type} dapat digunakan untuk mengetahui tipe data dari
suatu nilai dari variabel atau eskpresi.
\begin{pyconcode}
>>> type(1)
<type 'int'>
>>> type(1.)
<type 'float'>
>>> type(1. + 0j )
<type 'complex'>
>>> a = 3
>>> type(a)
<type 'int'>
\end{pyconcode}


Beberapa tipe data dapat dikonversi antara satu dengan yang lainnya.
\begin{pyconcode}
>>> a = 2
>>> type(a)
<class 'int'>
>>> b = float(a)
>>> b
2.0
>>> type(b)
<class 'float'>
>>> c = 2.3
>>> d = int(2.3)
>>> d
2
\end{pyconcode}

\section{Tipe data kontainer}
Python memiliki beberapa tipe data kontainer yang merupakan kumpulan
dari beberapa tipe objek.

\subsection{Lists}
List merupakan kumpulan objek yang memiliki urutan. Objek-objek ini
dapat memiliki tipe yang berbeda. Contoh
\begin{pyconcode}
>>> l = [1, 2, 3, 4, 5]
>>> type(l)
<type 'list'>
\end{pyconcode}

Kita dapat mengakses suatu objek dengan menggunan indeksnya.
\begin{pyconcode}
>>> l[2]
3
\end{pyconcode}

Indeks negatif dapat digunakan untuk mengakses dari akhir.
\begin{pyconcode}
>>> l[-1]
5
>>> l[-2]
4
\end{pyconcode}

Mirip dengan bahasa C/C++, indeks pada Python dimulai dari 0.

Operasi slicing dapat digunakan untuk mendapatkan sublist dari
suatu list (mirip dengan Matlab/Octave/Scilab).
\begin{pyconcode}
>>> l
[1, 2, 3, 4, 5]
>>> l[2:4]
[3, 4]
\end{pyconcode}

Perhatikan bawah \texttt{l[start:stop]} akan memberikan
elemen dengan indeks \texttt{i} di mana \texttt{start<= i < stop}.
Artinya, \texttt{l[start:stop]} akan memiliki
\texttt{(stop-start)} buah element.

Sintaks untuk slicing: \texttt{l[start:stop:stride]}.
Semua parameternya bersifat opsional.
\begin{pyconcode}
>>> l[3:]
[4, 5]
>>> l[:3]
[1, 2, 3]
>>> l[::2]
[1, 3, 5]
\end{pyconcode}

List bersifat \textit{mutable}, artinya nilainya dapat diubah.
\begin{pyconcode}
>>> l[0] = 28
>>> l
[28, 2, 3, 4, 5]
>>> l[2:4] = [3, 8] 
>>> l
[28, 2, 3, 8, 5]
\end{pyconcode}

Elemen-element pada suatu list dapat memiliki tipe yang berbeda.
\begin{pyconcode}
>>> l = [3, 2, 'hello']
>>> l
[3, 2, 'hello']
>>> l[1], l[2]
(2, 'hello')
\end{pyconcode}

Menambahkan dan mengurangi elemen pada suatu list.
\begin{pyconcode}
>>> l = [1, 2, 3, 4, 5]
>>> l.append(6)
>>> l
[1, 2, 3, 4, 5, 6]
>>> l.pop()
6
>>> l
[1, 2, 3, 4, 5]
>>> l.extend([6, 7])
>>> l
[1, 2, 3, 4, 5, 6, 7]
>>> l = l[:-2]
>>> l
[1, 2, 3, 4, 5]
\end{pyconcode}

Membalikkan urutan pada list.
\begin{pyconcode}
>>> r = l[::-1]
>>> r
[5, 4, 3, 2, 1]
\end{pyconcode}

Menyambung (\texttt{list}) dan mengurangi element pada list.
\begin{pyconcode}
>>> r + l
[5, 4, 3, 2, 1, 1, 2, 3, 4, 5]
>>> 2 * r
[5, 4, 3, 2, 1, 5, 4, 3, 2, 1]
\end{pyconcode}

Mengurutkan elemen-elemen pada list.
\begin{pyconcode}
>>> r.sort()
>>> r
[1, 2, 3, 4, 5]
\end{pyconcode}


Untuk perhitungan numerik dan komputasi performa tinggi, paket
\textsf{Numpy} lebih sering digunakan. Informasi mengenai \textsf{Numpy}
dapat ditemukan di \url{www.numpy.org}.


\subsection{Strings}

Tipe data string pada Python ditandai dengan menggunakan tanda
kutip tunggal (\texttt{"}) atau ganda (\texttt{"}). Tiga tanda kutip
(baik tunggal maupun ganda) dapat digunakan untuk string yang panjang.
\begin{pyconcode}
s = 'Hello, how are you?'
s = "Hi, what's up"
s = '''Hello, 
       how are you'''
s = """Hi,
   what's up?'''
\end{pyconcode}

Karakter baris baru adalah \verb|\n| dan karakter tab adalah.
\verb|\t|.

String dapat dianggap sebagai list dari karakter (huruf), artinya string
juga memiliki operasi yang sama dengan list.

Mengakses elemen lewat indeks.
\begin{pyconcode}
>>> a = "hello"
>>> a[0]
'h'
>>> a[1]
'e'
>>> a[-1]
'o'
\end{pyconcode}

Slicing.
\begin{pyconcode}
>>> a = "hello, world!"
>>> a[3:6]
'lo,'
>>> a[2:10:2]
'lo o'
>>> a[::3]
'hl r!'
\end{pyconcode}

Berbeda dengan list, string bersifat \textit{immutable}.
Kita tidak bisa mengubah karakter pada string, akan tetapi kita dapat
membuat string baru dari string awal.
String memiliki banyak metode yang dapat digunakan keperluan
manipulasi teks.

\subsubsection*{Tugas}

Baca dokumentasi Python mengenai string:
\url{https://docs.python.org/3/library/string.html}



\subsection{Dictionary/Kamus}

Tipe data kamus adalah tabel yang yang memetakan kunci ke suatu nilai.
\begin{pyconcode}
>>> tel = {'emmanuelle': 5752, 'sebastian': 5578}
>>> tel['francis'] = 5915 
>>> tel
{'sebastian': 5578, 'francis': 5915, 'emmanuelle': 5752}
>>> tel['sebastian']
5578
>>> tel.keys()
['sebastian', 'francis', 'emmanuelle']
>>> tel.values()
[5578, 5915, 5752]
>>> 'francis' in tel
True
\end{pyconcode}

Suatu kamus dapat memiliki kunci dan nilai yang memiliki tipe berbeda
\begin{pyconcode}
>>> d = {'a':1, 'b':2, 3:'hello'}
>>> d
{'a': 1, 3: 'hello', 'b': 2}
\end{pyconcode}

\subsection{Tupel}
Tupel dapat dianggap sebagai list \texttt{immutable}.
Elemen-elemen tupel ditulis dalam tanda kurung dan dipisahkan dengan
tanda koma.
\begin{pyconcode}
>>> t = 12345, 54321, 'hello!'
>>> t[0]
12345
>>> t
(12345, 54321, 'hello!')
>>> u = (0, 2)
\end{pyconcode}



\section{Kontrol alur program}

\subsection{Percabangan: \texttt{if/elif/else}}
  
\begin{pyconcode}
>>> a = 2
>>> if a < 0:
...     print("a bernilai negatif")
... else:
...     print("a bernilai positif")
... 
\end{pyconcode}


\textbf{Indentasi sangat penting pada Python.}

Jika Anda kesulitan atau tidak nyaman menuliskan di konsol, Anda
dapat menuliskannya di file dengan ekstensi \texttt{*.py}.
Anda tidak perlu menuliskan \verb|>>>| atau \verb|...|.

Contoh lain
\begin{pyconcode}
>>> a = 10 
>>> if a == 1:
...     print(1)
... elif a == 2:
...     print(2)
... else:
...     print('A lot')
... 
A lot
\end{pyconcode}


\subsection{Perulangan: \texttt{for/range}}

Perulangan dengan indeks
\begin{pyconcode}
>>> for i in range(4):
...     print(i)
... 
0
1
2
3
\end{pyconcode}

Iterasi terhadap nilai:
\begin{pyconcode}
>>> for word in ('cool', 'powerful', 'readable'):
...     print('Python is %s' % word)
...
Python is cool
Python is powerful
Python is readable
\end{pyconcode}

\subsection{Perulangan: \texttt{while/break/continue}}

\begin{pyconcode}
>>> a = 3;
>>> while a < 10:
...     a = a + 1
... 
>>> a
10
\end{pyconcode}

\texttt{break} dapat digunakan untuk
keluar dari loop \texttt{for} atau \texttt{while}
\begin{pyconcode}
>>> a = 10;
>>> while a > 0:
...     a = a - 2
...     if a < 3: break
... 
>>> a
2
\end{pyconcode}


\texttt{continue} dapat digunakan untuk meneruskan ke
iterasi selanjutnya (biasanya untuk melangkaui suatu iterasi)
\begin{pyconcode}
>>> a = [1, 0, 2, 4]
>>> for element in a:
...     if element == 0:
...         continue
...     print(1.0/element)
...     
1.0
0.5
0.25
\end{pyconcode}


\subsection{Iterasi pada string, list, dan dictionary}

Kita dapat melakukan iterasi terhadap string, list dan dictionary
\begin{pyconcode}
>>> vowels = 'aeiouy'
>>> for i in 'powerful':
...     if i in vowels:
...         print(i),
...         
...         
o e u
\end{pyconcode}

Contoh
\begin{pyconcode}
>>> message = "Hello how are you?"
>>> message.split() # mengembalikan list
['Hello', 'how', 'are', 'you?']
>>> for word in message.split():
...     print word
...     
Hello
how
are
you?
\end{pyconcode}

Contoh:
\begin{pyconcode}
>>> words = ('cool', 'powerful', 'readable')
>>> for i in range(0, len(words)):
....     print(i, words[i])
....     
....     
0 cool
1 powerful
2 readable
\end{pyconcode}

Fungsi \texttt{enumerate} juga dapat digunakan
\begin{pyconcode}
>>> words = ('cool', 'powerful', 'readable')
>>> for index, item in enumerate(words):
...     print index, item
...     
0 cool
1 powerful
2 readable
\end{pyconcode}


Loop pada dictionary dengan menggunakan \texttt{items()}
\begin{pyconcode}
>>> d = {"a" : 1, "b" : 1.2, "c" : 1j}
>>> for key, val in d.items():
        print('Key: %s has value: %s' % (key, val)
Key: a has value: 1
Key: c has value: 1j
Key: b has value: 1.2
\end{pyconcode}


\section{Mendefinisikan fungsi}

Sintaks dasar:
\begin{pyconcode}
>>> def test():
...     print("in test function")
...
>>> test()
in test function
\end{pyconcode}

\textbf{Ingat bahwa indentasi signifikan pada Python}

Fungsi dapat mengembalikan nilai
\begin{pyconcode}
>>> def disk_area(radius):
...     return 3.14 * radius * radius
...
>>> disk_area(1.5)
>>> 7.0649999999999995
\end{pyconcode}

Secara default, fungsi mengembalikan \texttt{None}.

Parameter fungsi pada Python dapat berupa parameter wajib yang sesuai
dengan posisinya pada definisi fungsi.

\begin{pyconcode}
>>> def double_it(x):
...     return x*2
... 
>>> double_it(3)
6
>>> double_it(3.2)
6.4
>>> double_it()
Traceback (most recent call last):
  File "<stdin>", line 1, in <module>
TypeError: double_it() missing 1 required positional argument: 'x'
>>> double_it(3,1)
Traceback (most recent call last):
  File "<stdin>", line 1, in <module>
TypeError: double_it() takes 1 positional argument but 2 were given
\end{pyconcode}

Argumen fungsi pada Python juga dapat bersifat
opsional, dengan menggunakan kata kunci atau nama argumen.
Dengan menggunakan argument kata kunci, kita dapat memberikan nilai
default pada fungsi.
\begin{pyconcode}
>>> def double_it(x=2):
...     return x * 2
...
>>> double_it()
>>> 4
>>> double_it(3)
>>> 6
\end{pyconcode}


\end{document}