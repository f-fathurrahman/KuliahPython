\section{Modul}

Seperti yang telah dijelaskan pada bagian awal, selain menggunakan interpreter (secara interaktif),
kita juga dapat menuliskan instruksi atau program Python pada sebuah file dengan ekstensi \texttt{.py}.
File ini sering juga disebut dengan \textit{script} yang biasanya berisikan instruksi-instruksi yang
relatif sederhana dan dapat dijalankan langsung oleh pengguna.

\subsection{\textit{Scripts}}

Sebagai contoh sederhana mari kita coba buat file dengan nama \texttt{halo.py}
dengan isi sebagai berikut.
\begin{pythoncode}
message = "Hello how are you?"
for word in message.split():
    print(word)
\end{pythoncode}

Untuk mengeksekusi \textit{script} ini, kita dapat menggunakan perintah:
\begin{bashcode}
python3 hello.py
\end{bashcode}

Kita juga dapat menjalankan \textit{script} ini pada saat kita menggunakan
IPython dengan menggunakan sintaks \texttt{\%run nama\_script.py}.
Contoh:
\begin{ipython3code}
In[1]: %run halo.py
\end{ipython3code}
Ketika perintah di atas dipanggil, kita script telah dieksekusi dan variabel dan
simbol yang ada pada script dapat kita akses pada IPython.

\textit{Scripts} juga dapat menerima argumen baris perintah.
Contoh pada file berikut ini, misalkan dengan nama \texttt{print\_args.py}
\begin{pythoncode}
import sys
print(sys.argv)
\end{pythoncode}
Kemudian pada terminal:
\begin{bashcode}
$ python3 print_args.py test arguments
['print_args.py', 'test', 'arguments']
\end{bashcode}


\subsection{Membuat modul}

Jika program kita sudah semakin besar dan kompleks, biasanya kita perlu membagi program
menjadi beberapa subprogram atau fungsi-fungsi, variabel, dan kelas
yang ditulis pada suatu file yang mirip
dengan \textit{script} dan disebut dengan modul.

Misalkan buat file dengan nama \texttt{modulku.py}
\begin{pythoncode}
def print_a():
    print("Ini adalah fungsi print_a dari modulku")

def print_b():
    print("Ini adalah fungsi print_b dari modulku")

c = 2
d = 2
\end{pythoncode}

Pada file ini kita mendefinisikan dua fungsi \verb|print_a| dan \verb|print_b|.
Misalkan kita ingin menggunakan fungsi \verb|print_a| dari interpreter.
Kita dapat mengeksekusi file/\textit{script} atau meng-importnya sebagai
modul.

\begin{pythoncode}
import modulku
modulku.print_a()
modulku.print_b()
\end{pythoncode}

Jika kita mengubah modul dan ingin mengimportnya pada suatu sesi yang sedang berjalan
kita hanya akan mendapatkan modul yang lama.
Untuk menggunakan module yang sudah terupdate kita harus melakukan \textit{reload}:
\begin{pythoncode}
import importlib
importlib.reload(modulku)
\end{pythoncode}

