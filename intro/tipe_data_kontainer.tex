\section{Tipe data kontainer}
Python memiliki beberapa tipe data kontainer yang merupakan kumpulan
dari beberapa tipe objek. Tipe data kontainer pada Python diantaranya
adalah:
\begin{itemize}
\item List
\item 
\end{itemize}

\subsection{Lists}
List merupakan kumpulan objek yang memiliki urutan. Objek-objek ini
dapat memiliki tipe yang berbeda. Contoh
\begin{pyconcode}
>>> l = [1, 2, 3, 4, 5]
>>> type(l)
<type 'list'>
\end{pyconcode}

Kita dapat mengakses suatu objek dengan menggunan indeksnya.
\begin{pyconcode}
>>> l[2]
3
\end{pyconcode}

Indeks negatif dapat digunakan untuk mengakses dari akhir.
\begin{pyconcode}
>>> l[-1]
5
>>> l[-2]
4
\end{pyconcode}

Mirip dengan bahasa C/C++, indeks pada Python dimulai dari 0.

Operasi slicing dapat digunakan untuk mendapatkan sublist dari
suatu list (mirip dengan Matlab/Octave/Scilab).
\begin{pyconcode}
>>> l
[1, 2, 3, 4, 5]
>>> l[2:4]
[3, 4]
\end{pyconcode}

Perhatikan bawah \texttt{l[start:stop]} akan memberikan
elemen dengan indeks \texttt{i} di mana \texttt{start<= i < stop}.
Artinya, \texttt{l[start:stop]} akan memiliki
\texttt{(stop-start)} buah element.

Sintaks untuk slicing: \texttt{l[start:stop:stride]}.
Semua parameternya bersifat opsional.
\begin{pyconcode}
>>> l[3:]
[4, 5]
>>> l[:3]
[1, 2, 3]
>>> l[::2]
[1, 3, 5]
\end{pyconcode}

List bersifat \textit{mutable}, artinya nilainya dapat diubah.
\begin{pyconcode}
>>> l[0] = 28
>>> l
[28, 2, 3, 4, 5]
>>> l[2:4] = [3, 8] 
>>> l
[28, 2, 3, 8, 5]
\end{pyconcode}

Elemen-element pada suatu list dapat memiliki tipe yang berbeda.
\begin{pyconcode}
>>> l = [3, 2, 'hello']
>>> l
[3, 2, 'hello']
>>> l[1], l[2]
(2, 'hello')
\end{pyconcode}

Menambahkan dan mengurangi elemen pada suatu list.
\begin{pyconcode}
>>> l = [1, 2, 3, 4, 5]
>>> l.append(6)
>>> l
[1, 2, 3, 4, 5, 6]
>>> l.pop()
6
>>> l
[1, 2, 3, 4, 5]
>>> l.extend([6, 7])
>>> l
[1, 2, 3, 4, 5, 6, 7]
>>> l = l[:-2]
>>> l
[1, 2, 3, 4, 5]
\end{pyconcode}

Membalikkan urutan pada list.
\begin{pyconcode}
>>> r = l[::-1]
>>> r
[5, 4, 3, 2, 1]
\end{pyconcode}

Menyambung (\texttt{list}) dan mengurangi element pada list.
\begin{pyconcode}
>>> r + l
[5, 4, 3, 2, 1, 1, 2, 3, 4, 5]
>>> 2 * r
[5, 4, 3, 2, 1, 5, 4, 3, 2, 1]
\end{pyconcode}

Mengurutkan elemen-elemen pada list.
\begin{pyconcode}
>>> r.sort()
>>> r
[1, 2, 3, 4, 5]
\end{pyconcode}


Untuk perhitungan numerik dan komputasi performa tinggi, paket
\textsf{Numpy} lebih sering digunakan. Informasi mengenai \textsf{Numpy}
dapat ditemukan di \url{www.numpy.org}.


\subsection{Strings}

Tipe data string pada Python ditandai dengan menggunakan tanda
kutip tunggal (\texttt{"}) atau ganda (\texttt{"}). Tiga tanda kutip
(baik tunggal maupun ganda) dapat digunakan untuk string yang panjang.
\begin{pythoncode}
s = 'Halo, nama saya Jojo'
s = "Halo, nama saya Joko"
s = '''Halo, 
       nama saya Jono'''
s = """Halo,
   Nama saya Joyo"""
\end{pythoncode}

Karakter baris baru adalah \verb|\n| dan karakter tab adalah \verb|\t|.

String dapat dianggap sebagai list dari karakter (huruf), artinya string
juga memiliki operasi yang sama dengan list.

Mengakses elemen lewat indeks.
\begin{pyconcode}
>>> a = "hello"
>>> a[0]
'h'
>>> a[1]
'e'
>>> a[-1]
'o'
\end{pyconcode}

Slicing.
\begin{pyconcode}
>>> a = "hello, world!"
>>> a[3:6]
'lo,'
>>> a[2:10:2]
'lo o'
>>> a[::3]
'hl r!'
\end{pyconcode}

Berbeda dengan list, string bersifat \textit{immutable}.
Kita tidak bisa mengubah karakter pada string, akan tetapi kita dapat
membuat string baru dari string awal.
String memiliki banyak metode yang dapat digunakan keperluan
manipulasi teks.

\begin{mdframed}[backgroundcolor=myframebg]
\textbf{Tugas}

Baca dokumentasi Python mengenai string:
\url{https://docs.python.org/3/library/string.html}
\end{mdframed}


\subsection{Dictionary/Kamus}

Tipe data kamus adalah tabel yang yang memetakan kunci ke suatu nilai.
\begin{pyconcode}
>>> tel = {'emmanuelle': 5752, 'sebastian': 5578}
>>> tel['francis'] = 5915 
>>> tel
{'sebastian': 5578, 'francis': 5915, 'emmanuelle': 5752}
>>> tel['sebastian']
5578
>>> tel.keys()
['sebastian', 'francis', 'emmanuelle']
>>> tel.values()
[5578, 5915, 5752]
>>> 'francis' in tel
True
\end{pyconcode}

Suatu kamus dapat memiliki kunci dan nilai yang memiliki tipe berbeda
\begin{pyconcode}
>>> d = {'a':1, 'b':2, 3:'hello'}
>>> d
{'a': 1, 3: 'hello', 'b': 2}
\end{pyconcode}


\subsection{Tupel}

Tupel dapat dianggap sebagai list yang tidak dapat dimodifikasi \texttt{immutable}.
Elemen-elemen tupel ditulis dalam tanda kurung dan dipisahkan dengan
tanda koma.
\begin{pyconcode}
>>> t = 12345, 54321, 'hello!'
>>> t[0]
12345
>>> t
(12345, 54321, 'hello!')
>>> u = (0, 2)
\end{pyconcode}