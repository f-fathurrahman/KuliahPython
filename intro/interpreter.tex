\section{Mengenal interpreter Python (konsol Python)}

Terdapat banyak pilihan untuk berinteraksi dengan interpreter 
atau konsol Python:
\begin{itemize}
\item Konsol default Python, dapat dijalankan dengan mengetikkan
\texttt{python} pada terminal.
\item IPython, merupakan interpreter Python dengan berbagai fitur tambahan
seperti \textit{tab-completion} dan \textit{syntax highlighting}.
IPython 
\item Jupyter qtconsole
\item Jupyter notebook
\end{itemize}

Tampilan awal konsol Python default:
\begin{textcode}
Python 3.6.6 |Anaconda custom (64-bit)| (default, Jun 28 2018, 17:14:51) 
[GCC 7.2.0] on linux
Type "help", "copyright", "credits" or "license" for more information.
>>>
\end{textcode}

Tampilan awal IPython
\begin{textcode}
Python 3.6.6 |Anaconda custom (64-bit)| (default, Jun 28 2018, 17:14:51) 
Type 'copyright', 'credits' or 'license' for more information
IPython 6.4.0 -- An enhanced Interactive Python. Type '?' for help.

In [1]:
\end{textcode}

Dalam praktikum ini Anda sebaiknya menggunakan IPython karena lebih
mudah digunakan. Jika IPython belum terpasang, Anda dapat menggunakan
konsol Python biasa.

Anda dapat menggunakan interpreter ini untuk menjalankan kode Python.
Kita akan mulai dengan contoh sederhana yaitu operasi aritmatika
sederhana.
\begin{pyconcode}
>>> 2 + 3
5
>>> 2 / 3
0.6666666666666666
\end{pyconcode}

Anda juga dapat mendefinisikan variabel dengan menggunakan operator
penugasan. Operator penugasan (\textit{assigment})
pada Python adalah tanda \texttt{=}.

\begin{pyconcode}
>>> a = 3
>>> b = 4.1
>>> a*b
12.299999999999999
>>> a - b
-1.0999999999999996
>>> a/b
0.7317073170731708
\end{pyconcode}

Fungsi \texttt{print} dapat digunakan menuliskan nilai suatu ekspresi dan variabe
\begin{pyconcode}
>>> x = 1.2
>>> print(x)
1.2
>>> print(x + 2)
3.2
\end{pyconcode}

Pada interpreter, kita dapat langsung mengetikkan variabel untuk mengetahui nilai
dari variabel atau suatu ekspresi (tanpa menggunakan fungsi \texttt{print}):
\begin{pyconcode}
>>> x
1.2
>>> x + 2
3.2
\end{pyconcode}

Python juga mendukung bilangan kompleks. Contoh:
\begin{pyconcode}
>>> a = 1.5 + 0.5j # tidak ada spasi sebelum j
>>> a.real
1.5
>>> a.imag
0.5
\end{pyconcode}

Contoh variabel boolean
\begin{pyconcode}
>>> 3 > 4
False
>>> test = (3 > 4)
>>> test
False
\end{pyconcode}

Fungsi \texttt{type} dapat digunakan untuk mengetahui tipe data dari
suatu nilai dari variabel atau eskpresi.
\begin{pyconcode}
>>> type(1)
<type 'int'>
>>> type(1.)
<type 'float'>
>>> type(1. + 0j )
<type 'complex'>
>>> a = 3
>>> type(a)
<type 'int'>
\end{pyconcode}

Berbeda dengan C/C++, suatu variabel tidak perlu dideklarasikan atau
tipenya sebelum digunakan. Python akan mendeduksi tipe variabel berdasarkan
nilai yang diberikan. Berbeda dengan C, variabel pada Python dapat merujuk
pada suatu nilai dengan tipe data yang berbeda.

\begin{pyconcode}
>>> z = 1.2
>>> type(z)
<class 'float'>
>>> z = 2
>>> type(z)
<class 'int'>
>>> z = 1 + 2j
>>> type(z)
<class 'complex'>
>>> z = 2 > 3
>>> type(z)
<class 'bool'>
\end{pyconcode}


\begin{mdframed}[backgroundcolor=myframebg]
\textbf{Tugas}

Carilah operator matematika yang didukung oleh Python.
Beberapa sumber berikut dapat Anda gunakan:
\begin{itemize}
\item \url{https://www.tutorialspoint.com/python/python_basic_operators.htm}
\item \url{https://en.wikibooks.org/wiki/Python_Programming/Basic_Math}
\end{itemize}
\end{mdframed}

